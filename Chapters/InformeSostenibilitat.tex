% Chapter Template

\chapter{Informe de sostenibilitat} % Main chapter title

\label{InformeSostenbilitat} % Change X to a consecutive number; for referencing this chapter elsewhere, use \ref{ChapterX}

Per tal d’evaluar la sostenibilitat del projecte a nivel econòmic, social i ambiental es fabrica la matriu de sostenibilitat, on s’atribueix a cada dimensió un
valor evaluant així cada aspecte.\\


\begin{table}[!h]
\begin{tabular}{|l|r|}
\hline
\textbf{Caràcter}  & \textbf{Valoració} \\\hline
Econòmic & 8 \\\hline
Social & 8 \\\hline
Ambiental & 7 \\\hline
\textbf{Total} & 23 \\\hline
\end{tabular}
\label{}
\caption{Matriu de sostenibilitat}
\end{table}

\section{Dimensió econòmica}
Per aquest projecte s’han avaluat detalladament els costos tant de recursos com
materials. No s’han tingut en compte els costos deguts a possibles ajustaments,
reparacions o actualitzacions durant el desenvolupament ja que es tracta d’un
projecte acadèmic. Sí que es cert que es podria desenvolupar el mateix projecte
en menys temps, però per fer-ho possible es necessitaria més recursos humans
i d’una major experiència professional. Pel que fa a la planificació es pot assegurar que les tasques en les que s’ha dedicat més hores són les més importants.
Podem dir que que el cost total del projecte ha estat ajustat al màxim tenint en
compte els costos de projectes de desenvolupament de software similars, i això
converteix el sistema en un producte competitiu.

\section{Dimensió social}

El producte resultant d’aquest projecte facilitarà molt la vida a tots els viatgers
que vulguin un suport per tal de millorar les condicions del viatge. La necessitat del productes és real ja que sense ell el viatger es veu més confús a l’hora
d’obtenir informació sobre l’entorn degut a que les aplicacions existents són
molt específiques i no s’adapten a la necessitat de l’usuari.

\clearpage
\section{Dimensió ambiental}
El desenvolupament del projecte provoca la utilització de diversos recursos
que afecte el medi ambient. En les diferents parts del projecte s’utilitza l’energia elèctrica com a font d’energia per poder treballar, el cost d’aquesta té un
impacte ambiental. En concret s’utilitzarà l’energia elèctrica necessària per alimentar l’ordinador que s’utilitza per a desenvolupar i el dispositiu mòbil on
es fan les proves. A més, s’ha de tenir en compte també que per utilitzar l’aplicació és necessari tenir connexió a internet, cosa que també suposa un cost
elèctric.\\

Amb aquest nou producte, podem dir que es minimitzaran les descàrregues
de múltiples aplicacions, cosa que disminueix el cost elèctric. A més, també
promou l’estalvi de paper en termes de guies de turisme i altres.
	
