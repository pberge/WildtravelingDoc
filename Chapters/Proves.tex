% Chapter Template

\chapter{Proves} % Main chapter title


\label{Proves} % Change X to a consecutive number; for referencing this chapter elsewhere, use \ref{ChapterX}

En aquest apartat es defineixen les proves que s'han fet sobre el sistema per tal d'assegurar un correcte funcionament.

\section{Proves de desenvolupament}

A continuació, es mostra per a cada cas d'ús, quines són les característiques que s'han de provar i el resultat esperat.

\begin{itemize}

\item{}\textbf{Registre i inici de sessió al sistema}

\begin{table}[!h]
\centering
\begin{tabular}{|T|T|}
\hline
\textbf{Característica} & \textbf{Resultat esperat} \\\hline
Connexió amb Google & Connexió satisfactòria\\\hline
\end{tabular}
\label{}
\caption{Proves \textit{Registrar-se al sistema}}
\end{table}

\item{}\textbf{Tancar sessió}

\begin{table}[!h]
\centering
\begin{tabular}{|T|T|}
\hline
\textbf{Característica} & \textbf{Resultat esperat} \\\hline
Des de la pantalla inicial, s'indica que es vol sortir. & Es mostra un diàleg de confirmació. Si s'accepta, es mostra la pantalla d'inici de sessió, si es cancel·la, es tanca el diàleg.\\\hline
\end{tabular}
\label{}
\caption{Proves \textit{Tancar sessió}}
\end{table}

\item{}\textbf{Crear viatge}

\begin{table}[!h]
\centering
\begin{tabular}{|T|T|}
\hline
\textbf{Característica} & \textbf{Resultat esperat} \\\hline
Algun camp és buit. & Es mostra un avís\\\hline
Les dades generals són correctes & Es mostra el diàleg de selecció d'estil.\\\hline
Es selecciona un estil de viatjar. & Si s'accepta, es mostra el diàleg d'introduir contacte d'emergència, si es cancel·la, es tanca el diàleg.\\\hline
Les dades del contacte d'emergència són correctes. & Es crea el viatge i es veu la pantalla de consultar el viatge.\\\hline
\end{tabular}
\label{}
\caption{Proves \textit{Crear viatge}}
\end{table}

\clearpage

\item{}\textbf{Consultar llista de viatges}

\begin{table}[!h]
\centering
\begin{tabular}{|T|T|}
\hline
\textbf{Característica} & \textbf{Resultat esperat} \\\hline
L'usuari no té viatges. & Es mostra la pantalla sense cap viatge.\\\hline
L'usuari té viatges creats. & Es mostra la llista de viatges.\\\hline
\end{tabular}
\label{}
\caption{Proves \textit{Consultar llista de viatges}}
\end{table}

\item{}\textbf{Consultar viatge}

\begin{table}[!h]
\centering
\begin{tabular}{|T|T|}
\hline
\textbf{Característica} & \textbf{Resultat esperat} \\\hline
Es prem el botó de tornar. & Es mostra la pantalla de llista de viatges.\\\hline
Es prem el botó de consultar despeses. & Es mostra la llista de despeses.\\\hline
Es prem el botó d'explorar. & Es mostra la pantalla de cerca de punts d'interès.\\\hline
Es prem el botó de previsió meteorològica. & Es mostra la previsió meteorològica.\\\hline
Es prem el botó d'emergències. & Es mostra la pantalla d'emergències.\\\hline
Es prem el botó de notes. & Es mostra la llista de notes.\\\hline
El viatge d'altres participants. & Es mostra la llista de participants.\\\hline
El viatge té tots els camps. & Es mostren les dades generals del viatge.\\\hline
\end{tabular}
\label{}
\caption{Proves \textit{Consultar viatge}}
\end{table}

\item{}\textbf{Consultar llista de despeses}

\begin{table}[!h]
\centering
\begin{tabular}{|T|T|}
\hline
\textbf{Característica} & \textbf{Resultat esperat} \\\hline
El viatge té despeses. & Es mostra la llista de despeses i el total acumulat.\\\hline
Es prem el botó de crear despesa. & Es mostra la pantalla de crear despesa.\\\hline
El viatge no té despeses. & Es mostra la pantalla sense cap despesa i el total cumulat a zero.\\\hline
\end{tabular}
\label{}
\caption{Proves \textit{Consultar llista de despeses}}
\end{table}

\item{}\textbf{Afegir nova despesa}

\begin{table}[!h]
\centering
\begin{tabular}{|T|T|}
\hline
\textbf{Característica} & \textbf{Resultat esperat} \\\hline
Algun camp és buit. & Es mostra un avís\\\hline
Les dades són correctes. & Es crea la despesa i es veu la pantalla de consultar la despesa.\\\hline
\end{tabular}
\label{}
\caption{Proves \textit{Afegir nova despesa}}
\end{table}

\clearpage

\item{}\textbf{Consultar despesa}

\begin{table}[!h]
\centering
\begin{tabular}{|T|T|}
\hline
\textbf{Característica} & \textbf{Resultat esperat} \\\hline
La despesa té tots els paràmetres. & Es mostren totes les dades de la despesa.\\\hline
La despesa té deutes associats. & Es mostra la llista de deutes.\\\hline
Es prem el botó de modificar despesa. & Es mostra la pantalla de modificar despesa.\\\hline
\end{tabular}
\label{}
\caption{Proves \textit{Consultar despesa}}
\end{table}

\item{}\textbf{Cobrar deute}

\begin{table}[!h]
\centering
\begin{tabular}{|T|T|}
\hline
\textbf{Característica} & \textbf{Resultat esperat} \\\hline
L'usuari prem mantingudament un deute. & Es mostra un diàleg de confirmació. Si s'accepta, es marca el deute com a pagat, si es cancel·la, es tanca el diàleg.\\\hline
\end{tabular}
\label{}
\caption{Proves \textit{Cobrar deute}}
\end{table}

\item{}\textbf{Modificar despesa}

\begin{table}[!h]
\centering
\begin{tabular}{|T|T|}
\hline
\textbf{Característica} & \textbf{Resultat esperat} \\\hline
L'usuari modifica algun paràmetre. & Es mostra el paràmetre modificat.\\\hline
L'usuari prem el botó de tornar. & Es guarden tots els canvis i es mostra un avís.\\\hline
\end{tabular}
\label{}
\caption{Proves \textit{Modificar despesa}}
\end{table}

\item{}\textbf{Eliminar despesa}

\begin{table}[!h]
\centering
\begin{tabular}{|T|T|}
\hline
\textbf{Característica} & \textbf{Resultat esperat} \\\hline
L'usuari prem el botó d'eliminar despesa. & Es mostra un diàleg de confirmació. Si s'accepta s'elimina la despesa i es mostra la llista de despeses, si es cancel·la es tanca el diàleg.\\\hline
\end{tabular}
\label{}
\caption{Proves \textit{Eliminar despesa}}
\end{table}

\item{}\textbf{Gestionar cerques}

\begin{table}[!h]
\centering
\begin{tabular}{|T|T|}
\hline
\textbf{Característica} & \textbf{Resultat esperat} \\\hline
L'usuari prem el botó de gestionar cerques. & Es mostra la pantalla de realitzar cerca.\\\hline
\end{tabular}
\label{}
\caption{Proves \textit{Gestionar cerques}}
\end{table}

\clearpage

\item{}\textbf{Realitzar cerca}

\begin{table}[!h]
\centering
\begin{tabular}{|T|T|}
\hline
\textbf{Característica} & \textbf{Resultat esperat} \\\hline
L'usuari selecciona un item a buscar. & Es mostra el nom de l'item a buscar.\\\hline
L'usuari prem el botó d'explorar. & Es realitza la cerca. Si l'usuari ha introduït localització es busca sobre la introduïda, si no n'ha introduït es busca sobre la localització real de l'usuari.Finalment es mostra la pantalla dels resultats amb tota la informació i ordenats per nivell de preu.\\\hline
No hi ha resultats de cerca & Es mostra un avís.\\\hline
\end{tabular}
\label{}
\caption{Proves \textit{Realitzar cerca}}
\end{table}

\item{}\textbf{Consultar ruta}

\begin{table}[!h]
\centering
\begin{tabular}{|T|T|}
\hline
\textbf{Característica} & \textbf{Resultat esperat} \\\hline
L'usuari selecciona un item de la llista de resultats. & Es mostra la pantalla de consultar ruta amb la informació sobre el mapa de la ruta a peu, la distància i el temps.\\\hline
L'usuari selecciona la ruta amb vehicle. & Es mostra la pantalla de consultar ruta amb la informació sobre el mapa de la ruta amb vehicle, la distància i el temps.\\\hline
L'usuari selecciona el canvi de sentit de la ruta & Es mostra la pantalla de consultar ruta amb la informació sobre la nova ruta.\\\hline
\end{tabular}
\label{}
\caption{Proves \textit{Consultar ruta}}
\end{table}

\item{}\textbf{Consultar previsió meteorològica}

\begin{table}[!h]
\centering
\begin{tabular}{|T|T|}
\hline
\textbf{Característica} & \textbf{Resultat esperat} \\\hline
L'usuari prem el botó de consultar previsió meteorològica. & Es mostra la informació de la previsió meteorològica la informació actualitzada.\\\hline
\end{tabular}
\label{}
\caption{Proves \textit{Consultar previsió meteorològica}}
\end{table}

\item{}\textbf{Gestionar emergències}

\begin{table}[!h]
\centering
\begin{tabular}{|T|T|}
\hline
\textbf{Característica} & \textbf{Resultat esperat} \\\hline
L'usuari prem el botó de gestionar emergències. & Es mostra la pantalla de gestionar emergències.\\\hline
\end{tabular}
\label{}
\caption{Proves \textit{Gestionar emergències}}
\end{table}

\clearpage

\item{}\textbf{Trucar telèfon d'emergència}

\begin{table}[!h]
\centering
\begin{tabular}{|T|T|}
\hline
\textbf{Característica} & \textbf{Resultat esperat} \\\hline
L'usuari prem el botó de trucar al telèfon d'emergències. & S'inicia una trucada amb el telèfon d'emergències.\\\hline
\end{tabular}
\label{}
\caption{Proves \textit{Trucar telèfon d'emergències}}
\end{table}

\item{}\textbf{Trucar telèfon de contacte}

\begin{table}[!h]
\centering
\begin{tabular}{|T|T|}
\hline
\textbf{Característica} & \textbf{Resultat esperat} \\\hline
L'usuari prem el botó de trucar al telèfon de contacte. & S'inicia una trucada amb el telèfon de contacte.\\\hline
\end{tabular}
\label{}
\caption{Proves \textit{Trucar telèfon de contacte}}
\end{table}

\item{}\textbf{Consultar ruta a l'hospital}

\begin{table}[!h]
\centering
\begin{tabular}{|T|T|}
\hline
\textbf{Característica} & \textbf{Resultat esperat} \\\hline
L'usuari prem el botó de consultar ruta a l'hospital. & Es mostra la pantalla de consultar ruta amb la informació sobre el mapa de la ruta a peu a l'hospital més proper, la distància i el temps.\\\hline
\end{tabular}
\label{}
\caption{Proves \textit{Consultar ruta a l'hospital}}
\end{table}

\item{}\textbf{Consultar llista de notes}

\begin{table}[!h]
\centering
\begin{tabular}{|T|T|}
\hline
\textbf{Característica} & \textbf{Resultat esperat} \\\hline
El viatge no té notes. & Es mostra la pantalla de llistar notes sense cap nota.\\\hline
El viatge conté notes. & Es mostra la llista de notes.\\\hline
\end{tabular}
\label{}
\caption{Proves \textit{Consultar llista de notes}}
\end{table}

\item{}\textbf{Crear nota}

\begin{table}[!h]
\centering
\begin{tabular}{|T|T|}
\hline
\textbf{Característica} & \textbf{Resultat esperat} \\\hline
Algun camp és buit. & Es mostra un avís.\\\hline
L'usuari selecciona un color. & Es tenyeix el fons de la nota amb el color seleccionat.\\\hline
L'usuari comparteix la nota amb algun participant del viatge. & Es mostra el nom i el correu electrònic del participant amb qui s'ha compartit la nota.\\\hline
L'usuari prem el botó de crear la nota & Es crea la nota, s'envia un correu amb el contingut de la nota als participants amb qui s'ha compartit la nota i es veu la pantalla de consultar la nota creada.\\\hline
\end{tabular}
\label{}
\caption{Proves \textit{Crear nota}}
\end{table}

\clearpage

\item{}\textbf{Consultar nota}

\begin{table}[!h]
\centering
\begin{tabular}{|T|T|}
\hline
\textbf{Característica} & \textbf{Resultat esperat} \\\hline
La nota està compartida. & Es mostra la llista de viatgers amb qui s'ha compartit la nota.\\\hline
L'usuari selecciona un color. & Es tenyeix el fons de la nota amb el color seleccionat.\\\hline
\end{tabular}
\label{}
\caption{Proves \textit{Consultar nota}}
\end{table}

\item{}\textbf{Eliminar nota}

\begin{table}[!h]
\centering
\begin{tabular}{|T|T|}
\hline
\textbf{Característica} & \textbf{Resultat esperat} \\\hline
L'usuari prem el botó d'eliminar nota & Es mostra un diàleg de confirmació. Si s'accepta s'elimina la nota i es mostra la llista de notes, si es cancel·la es tanca el diàleg.\\\hline
\end{tabular}
\label{}
\caption{Proves \textit{Eliminar nota}}
\end{table}

\end{itemize}

Per a tots els casos d'ús, es comprova que les dades apareguin a la pantalla amb el format correcte i sense errors. A més, en molts punts es guarden dades a la base de dades remota, es comprova també que aquest emmagatzematge sigui correcte.

\section{Proves finals}

Les proves finals són el pas previ al llançament de l'aplicació al mercat. Per aquest motiu, s'ha demanat a un conjunt d'usuaris que utilitzin l'aplicació en els seus viatges. El \textit{feedback} rebut per part dels usuaris després de provar WildTraveling es pot resumir de la següent manera:
\begin{itemize}
\item{}Els usuaris opinen que l'aplicació és molt fàcil d'utilitzar i la interfície és molt elegant i senzilla.
\item{}Els usuaris no han tingut problemes en identificar les funcionalitats.
\item{}En la funcionalitat de cercar punts d'interès, alguns usuaris han trobat que la selecció de la localització no queda clara, ja que si el camp es deixa buit significa que s'utilitza la localització real de l'usuari.
\item{}A l'hora de crear una nova despesa, alguns usuaris han senyalat que si el viatge que dus a terme té molts participants, quan s'afegeixen deutes a la despesa estaria bé una opció que permeti afegir la mateixa quantitat a tots els participants.
\item{}Els usuaris han comprovat que l'aplicació és ràpida i no té problemes d'execució.
\end{itemize}

La valoració dels usuaris es considera molt bona, ja que han donat molta importància a la facilitat d'ús d'aquesta i a un disseny elegant i senzill. A més els petits inconvenients que han pogut identificar comporten canvis molt concrets en el sistema que es poden solucionar fàcilment de cara a pròximes actualitzacions i, en cap cas, aquests inconvenients comprometen el correcte funcionament del software.