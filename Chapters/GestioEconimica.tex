% Chapter Template

\chapter{Gestió Econòmical} % Main chapter title

\label{GestioEconomica} % Change X to a consecutive number; for referencing this chapter elsewhere, use \ref{ChapterX}

\section{Identificació i estimació dels costos}
En aquest punt es fa una estimació dels elements que formen part del pressupost del projecte tenint en compre els recursos que s’utilitzen.

\begin{itemize}
\item{}\textbf{Recursos humans}\\
Aquest projecte el desenvolupa únicament una persona, per aquest motiu
es necessari adoptar diferents rols. A continuació es mostra una taula amb
el preu que cobra per hora cada perfil implicat en el projecte, les hores
estimades que durà a terme cada rol i el cost total.



\begin{table}[!h]
\begin{tabular}{|c|c|c|c|c|}
\hline
\textbf{Recursos}  & \textbf{Rol} & \textbf{\euro/h} & \textbf{Estimació en hores} & \textbf{Total estimat} \\ \hline
 & Cap de projecte & 45 \euro/h & 160 h & 7.200 \euro \\ \cline {2-5}
\multicolumn{ 1}{|l|}{} & Analista & 30 \euro/h & 45 h & 1.350 \euro \\ \cline{ 2- 5}
Pere Bergé & Dissenyador & 30 \euro/h & 70 h & 2.100 \euro \\ \cline{ 2- 5}
\multicolumn{ 1}{|l|}{} & Programador & 25 \euro/h & 104 h & 2.600 \euro \\ \cline{ 2- 5}
\multicolumn{ 1}{|l|}{} & Tester & 15 \euro/h & 30 h & 450 \euro \\ \hline
\multicolumn{ 1}{|l|}{} &   &  & 409 h & 13.700 \euro \\ \hline
\end{tabular}
\label{}
\caption{Estimació cost recursos humans}
\end{table}

\item{}\textbf{Software}\\
Per dur a terme aquest projecte són necessàries diverses eines de software,
la majoria d’elles gratuïtes.

\begin{table}[!h]
\begin{tabular}{|c|c|c|c|}
\hline
\textbf{Producte}  & \textbf{Cost} & \textbf{Vida útil} & \textbf{Amortització total(5 mesos)}  \\ \hline
Windows 7 &  0\euro &   - & 0\euro \\\hline
Microssoft Office 2013 &  119\euro &  4 anys & 12,40\euro \\\hline
Android Studio 1.5.1 &  0\euro &   - & 0\euro \\\hline
Mozilla Firefox &  0\euro &   - & 0\euro \\\hline
Google Drive &  0\euro &   - & 0\euro \\\hline
Trello &  0\euro &   - & 0\euro \\\hline
SmartSheet &  0\euro &   - & 0\euro \\\hline
Adobe Photoshop &  12,09\euro/mes & 5 mesos & 60,45\euro \\\hline
Adobe Reader IX &  0\euro &   - & 0\euro \\\hline
Github &  0\euro &   - & 0\euro \\\hline
\textbf{Total} &  &    & 72,85\euro \\\hline
\end{tabular}
\label{}
\caption{Estimació cost recursos software}
\end{table}

\clearpage

\item{}\textbf{Hardware}

\begin{table}[!h]
\begin{tabular}{|c|c|c|c|}
\hline
\textbf{Producte}  & \textbf{Cost} & \textbf{Vida útil} & \textbf{Amortització total(5 mesos)}  \\ \hline
Ordinador portàtil &  600\euro/mes & 5 anys & 50\euro \\\hline
Smartphone &  190\euro &  3 anys & 26,39\euro \\\hline
\textbf{Total} &  &    & 76,39\euro \\\hline
\end{tabular}
\label{}
\caption{Estimació cost recursos hardware}
\end{table}

\item{}\textbf{Despeses generals}\\
En aquest apartat es tenen en compte les despeses associades a la realització del projecte com poden ser la energia elèctrica consumida o l’accés a internet.


\begin{table}[!h]
\begin{tabular}{|c|c|c|c|}
\hline
\textbf{Producte}  & \textbf{Cost} & \textbf{Periode} & \textbf{Total estimat}  \\ \hline
Energia elèctrica &  0,15\euro/kWh*0,15kW & 409 h & 9,20\euro \\\hline
Llicència desenvolupador \textit{Android} &  19\euro &  Indefinit & 19\euro \\\hline
Accés a internet &  25\euro &  5 mesos & 125\euro \\\hline
\textbf{Total} &  &    & 153,2\euro \\\hline
\end{tabular}
\label{}
\caption{Estimació cost despeses generals}
\end{table}

\item{}\textbf{Costos directes}\\
En aquest apartat s’agrupen tots els costos directes que suposa el desenvolupament d’aquest projecte.

\begin{table}[!h]
\begin{tabular}{|l|c|}
\hline
\textbf{Concepte}  & \textbf{Cost} \\\hline
Recursos humans & 13.700\euro \\\hline
Software & 73\euro \\\hline
Hardware & 77\euro \\\hline
\textbf{Total} & 13.850\euro \\\hline
\end{tabular}
\label{}
\caption{Estimació costos directes}
\end{table}

\item{}\textbf{Costos d’imprevistos}\\
L’únic cost imprevist que podria sorgir és una fallada en l’ordinador o l’smartphone que s’utilitzen pel desenvolupament del projecte. En cas de fallada s’hauria d’afegir el cost de reparació, aproximadament uns 150 \euro en cas de l’ordinador i 60 \euro en cas del dispositiu Android. Durant el transcurs de la reparació s’utilitzaria un segon aparell per tal de no bloquejar el desenvolupament del projecte. La probabilitat d’error d’ambdós aparells és d’un 15 \%.

\begin{table}[!h]
\begin{tabular}{|l|c|c|c|}
\hline
\textbf{Producte}  & \textbf{Cost Reparació} & \textbf{Probabilitat d'error} & \textbf{Total estimat}  \\ \hline
Ordinador & 150\euro & 15\% & 22,5\euro \\\hline
Dispositiu \textit{Android} &  60\euro &  15\% & 9\euro \\\hline
\textbf{Total} & -  & -    & 31,5\euro \\\hline
\end{tabular}
\label{}
\caption{Estimació cost d'imprevistos}
\end{table}

\clearpage

\item{}\textbf{Cost total del projecte}
\begin{table}[!h]
\begin{tabular}{|l|c|}
\hline
\textbf{Concepte}  & \textbf{Cost} \\\hline
Costos directes & 13.850\euro \\\hline
Despeses generals & 153,2\euro \\\hline
Cost d'imprevistos & 31,5\euro \\\hline
Contingències (5\%) & 701,7\euro \\\hline
\textbf{Total} & 14.740\euro \\\hline
\end{tabular}
\label{}
\caption{Estimació cost total del projecte}
\end{table}

\end{itemize}

\section{Control de gestió}
Per al control de gestió es planteja omplir unes taules amb les hores reals que dur a terme cadascun dels rols. Cada dia s’han d’actualitzar i d’aquesta manera es pot portar un control i veure possibles desviacions en la planificació.

\begin{table}[!h]
\begin{tabular}{|c|c|c|c|c|c|}
\hline
\textbf{Dia}  & \textbf{Cap de Projecte} & \textbf{Analista} & \textbf{Dissenyador} & \textbf{Programador} & \textbf{Tester}  \\ \hline
28/02/17 & 3h & - & - & - & - \\\hline
\end{tabular}
\label{}
\caption{Control d'hores reals}
\end{table}

Al final del projecte es farà un recompte de les hores reals i es compararan amb les estimades, d’aquesta manera es podrà analitzar la precisió de l’estimació. Les dades es presentaran en una taula com la que es mostra a continuació.

\begin{table}[!h]
\begin{tabular}{|c|c|c|c|c|c|}
\hline
\textbf{Rol}  & \textbf{Hores estimades} & \textbf{Cost estimat} & \textbf{Hores reals} & \textbf{Cost real}   \\ \hline
Cap de projecte & 160h & 7.200\euro & 170h & 7.650\euro \\\hline
Analista & 45h & 1.350\euro & 65h & 1.900\euro \\\hline
\end{tabular} 
\label{}
\caption{Comparació cost estimat i cost real}
\end{table}
